% tea.tex - An article about making tea
% Licensed under the Creative Commons Attributon-NonCommercial-ShareAlike 3.0 Unported (CC BY-NC-SA 3.0)
% license. 
%
% A summary (and full legalese if required) of the license can be found at:
% http://creativecommons.org/licenses/by-nc-sa/3.0/

\documentclass{article}
\usepackage{times}

\begin{document}

\title{Making Tea}
\author{Paul Lenton\\
 Flat 3, 139 Cheriton Road\\
 Folkestone\\
 Kent\\
 CT19 5HE\\
 \texttt{lentonp@gmail.com}}
\date{10th August, 2013}
\maketitle


\begin{abstract}
A document describing the technical requirements and instructions to make a cup of tea
\end{abstract}

\section{Introduction}

This section's content is designed to introduce making tea.

\bigskip\noindent
A cup of tea - made from boiled leaves of the {\it camellia sinensis} plant - is a typically
British beverage, enjoyed by millions of people across the UK, and the rest of the world.

\bigskip\noindent
Drinking tea holds many health benefits; it is lower in caffeine than coffee, contains a large
number of anti-oxidants (which have been shown to have some effect fighting {\it free radicals},
potentially cancer-causing agents in the body) and when cooled can be used to treat a variety of
skin conditions including sunburn.

\bigskip\noindent
There is much argument about what constitutes the perfect cup of tea, with arguments given for
green teas, black teas, mixed teas and fruit teas, as well as varying opinions on whether milk should
be added before or after the hot water is added.

Despite the arguments however, one thing that can be easily agreed upon is that a cup of tea (should
one enjoy the flavour) can be an uplifting and refreshing beverage.

\section{Making Tea: Pre-requisites}

There are many pre-requisites to making a {\bf good cup of tea}, but here we are not interested in the
differing opinions of what constitutes the {\bf best} cup of tea, but rather the essential requirements
for making a cup of tea. 

Once the initial process has been learnt and perfected, minor alterations can be made to ensure a cup of
tea more in fitting with your personal preferences. 

\subsection{Requirements}

To make a cup of tea, there are some essential items that you will need before you begin. In order to
follow along, please ensure that you have the following to hand:
\bigskip
\begin{itemize}
\item Loose leaf tea (black tea - you can choose your favourite flavour)
\item Teapot
\item Teaspoon
\item Tea-strainer (a fine-meshed sieve can be used in lieu of a tea-strainer, although this
is not ideal)
\item Kettle - this must be of the kind that switches off when the water has boiled
(if electric) or otherwise alerts you that the water is boiling
\item Access to a water supply
\item Access to an electricity supply (if using an electric kettle)
\item Access to a hob (if using a non-electric kettle)
\item Teacup (a coffee-mug or similar can be used in lieu of a teacup)
\item Milk-jug (or a similar small container)
\item Milk
\item Optional - Sugar
\item A space where you can make your tea
\end{itemize}
\noindent
Once you have all of the above to hand, you are ready to begin making a cup of tea.

\section{Preparation}

There is a small amount of prep-work that goes into making a cup of tea. It is important that all
steps are followed in order to ensure that you are able to make a cup of tea. Failure to carry out
some of these steps may result in a poor, undrinkable, cup of tea.

Whilst this document is not concerned with making a good cup of tea {\it per se} , should the tea be
undrinkable once all steps have been completed, it would not be fair to say that you had made a cup of
tea. Please therefore ensure that {\bf all} preparation steps are followed.

\subsection{Getting Set}

\begin{itemize}
\item Ensure that your teapot is clean, and otherwise empty. Leftover tea leaves will result in a poor (or worse, undrinkable) cup of tea
\item Ensure that all of your items are stored upright, on a safe and secure worktop. Failure to make a 
cup of tea could be down to something as simple as your teapot having been stored upside down
\item Ensure that you will not become distracted during the course of making your tea, leaving tea leaves 
to brew for too long, or leaving your boiling water to go cold for example will produce an undrinkable cup
of tea
\item Ensure your milk is fresh and has been appropriately refrigerated. Whilst some leeway is given regarding the freshness of milk when making tea, milk that is no longer fresh will render your tea undrinkable
\end{itemize}
\noindent
Preparation now complete, it's time to start making the cup of tea.

\section{Making Tea}
\subsection{Measuring Your Tea Leaves}
It is important to ensure that you do not use too much or too little tea. Too much tea will make a very strong and bitter drink, too little will add no flavour to the mix and you will be left with essentially boiling water.

\bigskip\noindent
The correct measurements for tea are as follows:

\bigskip\noindent
Using a black tea, you will require 2-3 teaspoons of loose-leaf tea per cup (a normal teapot will hold 6-7 cups of tea, although there will normally be an indication on your teapot of how many cups it will hold), plus 2-3 teaspoons for the pot.

\bigskip\noindent
When tea is boiled it will lose some of its natural flavour owing to chemical bonds being broken by the boiling water. In order to make up for this lost flavour, you add the 2-3 teaspoons for the pot, ensuring that after brewing, your tea is still at the correct strength.

Once you have calculated how many teaspoons of tea are required, please make a mental note of this number as it will be required in the next step.

\subsection{Adding Tea To The Pot}
When adding tea to the pot, each teaspoon's worth of tea leaves should be approximately equal. The teaspoon should be slightly heaped every time.

\bigskip\noindent
Add as many slightly heaped teaspoons of tea leaves to your pot as noted in the previous step. Once you have added your tea leaves, you can put your loose leaf tea away - it is no longer required in the process and will help to keep your workspace tidy.

\subsection{Boiling Water}
Assuming no prior knowledge of how to boil water in preparation for making tea, please carefully read the following instructions for full guidance.
\begin{itemize}
\item Using your water supply, fill the kettle to it's maximum safe capacity. This should be given on the side of the kettle, however a good rule-of-thumb is that your kettle should be no more than 75\% filled with water
\item If you are using an electric kettle, please read section 4.3.1, otherwise if you are using a kettle which is to be boiled on the hob, please read section 4.3.2
\end{itemize}
\subsubsection{Electric Kettles}
\begin{itemize}
\item Locate the nearest electrical outlet socket and ensure that your kettle is plugged in and the plug socket is powered on
\item Ensure that the kettle is similarly connected to the plug. Some kettles have a plug-in lead, others have a docking station
\item After ensuring your kettle is properly connected, flick the switch on your kettle
\item The adage that a watched kettle never boils is untrue. If you are unsure of approximately how long it will take to boil the kettle, please ensure that you watch the switch until it goes from the depressed position to the upwards position
\item Please move to section 4.4
\end{itemize}
\subsubsection{Non-Electric Kettles}
\begin{itemize}
\item Set your hob to a sufficient temperature to boil your kettle full of water. Whilst any temperature will eventually boil the water in your kettle, a temperature setting at around 50\% of its maximum provides a good time/safety mixture
\item Place the kettle which you have filled with water on the hob in accordance with the hob manufacturer's guidelines
\item Your kettle will give some kind of audible indication that it has boiled - normally this will be a high-pitched whistle, although if you are unsure please check with the kettle manufacturer
\item Once the kettle has boiled, remove it from the hob, turn the hob off and place the kettle somewhere safe and heatproof
\item Please move to section 4.4 
\end{itemize}
\subsection{Hot, not boiling, water}
To make tea, you will need hot but not boiling water. The previous step made sure you boiled the water. This is because boiling the water will help to kill off any microbes that lived in the water. If you heated the water without boiling it, you may not kill all of the microbes in the water.

As you are now left with boiling water, you will need to leave your kettle to stand for 5 minutes, leaving it time to cool down.

Once the kettle has stood for 5 minutes, you will be left with hot, not boiling, water.

\subsection{Adding Water To The Pot}
With the teapot lid off, slowly pour your hot water into the large opening at the top of the pot. You should try to fill the pot to about 8/10 of it's maximum capacity.

When you have sufficiently filled your teapot, you can put your kettle to one side, as you will not need it again during this process.

\subsection{Stirring The Tea}
With the lid off your teapot, stir the contents with your teaspoon. Stir 8-10 times clockwise, then the same number of times anti-clockwise. This will help to move the tea-leaves around the pot, allowing maximum diffusion of flavour. Once this has been done, replace the teapot's lid.

Leave the tea to stand in the pot for 3-5 minutes.

You can now put your teaspoon to one side. Please note that you will require the spoon later in the process.

\subsection{Making Tea}
You are now almost at the stage of finally making tea. For this step you will require your tea-cup (or other suitable container), your milk, your tea strainer and - if you have opted to include it, your sugar.
\begin{itemize}
\item Set your tea-cup down and add milk from your milk jug to the cup. You should fill 1/10 of the cup with milk. {\bf N.B. At your discretion you may skip this step at this stage and return to add the milk after all other "Making Tea" sections have been completed.}
\item Rest your tea strainer over your tea-cup. At this point, if you are using a fine-meshed sieve, you will have to sit it on your teacup. Try resting the middle of the sieve over the cup for best balance
\item Resting one finger on the teapot's lid, and firmly grasping the handle, begin pouring tea slowly into the cup through the tea strainer.
\item Stop pouring when your cup is 8/10 full
\item Set your teapot to one side and remove the tea-strainer from the top of your cup
\item {\it Optional step - adding sugar: Should you wish to take your tea with sugar, at this point, add as many teaspoons of sugar as you require to the cup}
\item Using your teaspoon, stir your tea 3-4 clockwise, then the same number of time anti-clockwise. You may now set your teaspoon to one side as you will no longer require it
\end{itemize}

\section{Success}
If you have followed all of the above steps, you should now have a drinkable cup of tea. 

Tea can be enjoyed as a refreshing beverage by itself, or served with a collection of sandwiches in the late afternoon with friends. It can be enjoyed with biscuits, with chocolate, or with cake. 

There are many different types of tea, each with their own caffeine content and flavours. Find one you enjoy for different times of the day, safe in the knowledge that with this technical guide, you will always be able to make a cup of tea.

You can now join the millions of people in the UK, and around the world, who make and drink tea on a regular basis.

\end{document}